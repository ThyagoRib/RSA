\documentclass[a4paper, 12pt]{article}

\usepackage[brazil]{babel}
\usepackage[utf8]{inputenc}
\usepackage{indentfirst}
\usepackage{tocbibind}

\begin{document}
	\begin{titlepage}
		\begin{center}
			\Huge{Universidade Federal de Alagoas}\\
			\large{Laboratório de Computação Científica e Visualização}\\ 
			\vspace{4 cm}
			\textbf{\LARGE{Projeto: Módulo RSA}}\\
			\vspace{3,5cm}
		\end{center}
		\begin{center}
			Thyago André Nunes Ribeiro
		\end{center}
		\begin{center}
			\vspace{\fill}
			Abril\\2019
		\end{center}
	\end{titlepage}
    
	\begin{titlepage}
		\begin{center}
			\Huge{Universidade Federal de Alagoas}\\
			\large{Laboratório de Computação Científica e Visualização}\\ 
			\vspace{4 cm}
			\textbf{\LARGE{Projeto: Módulo RSA}}
		\end{center}
		\vspace{1,5cm}
		\begin{flushright}
			\begin{list}{}{
				\setlength{\leftmargin}{4.5cm}
				\setlength{\rightmargin}{0cm}
				\setlength{\labelwidth}{0pt}
				\setlength{\labelsep}{\leftmargin}}
				\item Relatório de desenvolvimento de um módulo de criptografia RSA, apresentado como parte do processo seletivo e treinamento do Laboratório de Computação Científica e Visualização.
			\end{list}
			\vspace{3 cm}
			\begin{center}
				Thyago André Nunes Ribeiro
			\end{center}
		\end{flushright}
		\vspace{1cm}
		\begin{center}
			\vspace{\fill}
			Abril\\2019
		\end{center}
	\end{titlepage}
    
	\pagenumbering{arabic}
	
	\tableofcontents
	\newpage
	\section{Apresentação}
		\vspace{0,5 cm}
		\par Como parte do processo de treinamento foi solicitado o desenvolvimento de um módulo de criptografia RSA, utilizando a linguagem de programação Python.
		\\
		\par O projeto consiste, basicamente, de um módulo RSA que gera, aleatoriamente, ou recebe do usuário um conjunto de chaves públicas e privadas e executa as operações de criptografia e descriptografia de mensagens. Esse módulo deve ser controlado por uma classe Cliente que, antes de garantir o acesso do usuário ao módulo, faz o gerenciamento de cadastro e autenticação de usuários.
		\\
		\par Ao fim da execução do programa, a classe Cliente ainda é responsável por criptografar e armazenar os dados dos usuários cadastrados em um arquivo externo, utilizado, em uma nova execução, para pré-carregar os usuários no sistema.
	
	\newpage
	\section{Metodologia de implementação}
		\vspace{0,5 cm}
		\par A implementação do projeto dividiu-se em duas etapas. Na primeira, foram implementadas as classes e funções referentes ao módulo de criptografia RSA. Nessa fase, foram aplicados os conceitos apresentados em \cite{evaristo} referentes ao método de aplicação da criptografia RSA e as devidas funções e conhecimentos necessários. 
		\\
		\par Com isso, ao final do processo de implementação, foram obtidas as classes: RSA, responsável pela criação e gerenciamento das chaves públicas e privadas do módulo, bem como a garantia da validade de tais chaves, assim como o gerenciamento das chamadas dos métodos de codificação e decodificação da mensagem. A classe Encrypter, responsável pela pré-codificação da mensagem e a aplicação do algoritmo de criptografia RSA. E, por fim, a classe Decrypter, responsável pela aplicação do algoritmo de decodificação da criptografia RSA. Também foi criada uma biblioteca de funções lógicas e matemáticas, com métodos comuns aos processos de codificação e decodificação.
		\\
		\par A segunda parte da implementação consistiu no desenvolvimento de uma classe Cliente, cuja principal funcionalidade é o gerenciamento de usuários do sistema. Essa classe é capaz de realizar operações de adição e remoção de usuários do sistema, assim como a verificação da existência de usuários cadastrados e sua devida autenticação no sistema. Além disso, a classe Cliente é responsável pelos processos de pré-carregamento da lista de usuários cadastrados na inicialização do sistema, bem como a exportação de tal lista para um arquivo externo antes de encerrar o programa.
		\\
		\par Ao longo de todo o processo de implementação, foram realizados os devidos tratamentos de erros, segundo às especificações fornecidas para o projeto, bem como a documentação de todas as classes, módulos e métodos. Comentários adicionais foram adicionados ao longo do código a fim de esclarecer a compreensão de trechos de sua execução.
	
	\newpage
	\section{Conclusão}
		\par Analisadas e implementadas as especificações do projeto, obtivemos um sistema funcional, capaz de gerenciar usuários e seu devido acesso ao módulo de criptografia RSA, bem como realizar os devidos procedimentos de codificação e decodificação de mensagens.
	
	\newpage
	\bibliographystyle{abbrv}
	\bibliography{referencia}

 \end{document}